\documentclass[PMO,authoryear,toc]{lsstdoc}
\usepackage{epigraph}
\usepackage{csquotes}
\input{meta}

% Package imports go here.

% Local commands go here.

%If you want glossaries
%\input{aglossary.tex}
%\makeglossaries

\title{Access Control}

% Optional subtitle
% \setDocSubtitle{A subtitle}

\author{%
Joshua Hoblitt
}

\setDocRef{ITTN-068}
\setDocUpstreamLocation{\url{https://github.com/lsst-it/ittn-068}}

\date{\vcsDate}

% Optional: name of the document's curator
% \setDocCurator{The Curator of this Document}

\setDocAbstract{%
Access control policies
}

% Change history defined here.
% Order: oldest first.
% Fields: VERSION, DATE, DESCRIPTION, OWNER NAME.
% See LPM-51 for version number policy.
\setDocChangeRecord{%
  \addtohist{1}{YYYY-MM-DD}{Unreleased.}{Joshua Hoblitt}
}


\begin{document}

% Create the title page.
\maketitle
% Frequently for a technote we do not want a title page  uncomment this to remove the title page and changelog.
% use \mkshorttitle to remove the extra pages

% ADD CONTENT HERE
% You can also use the \input command to include several content files.

\epigraph{Anybody who has an identity problem had better wise up and get with the
program!}{\textit{Jack Handey}}

\section{Introduction}\label{sec:intro}

applies primarily to information technology related infrastructure for both electronic and physical access.

Conforms to NIST 800-53

The concept of "Least Privilege" is employeed per 800-53r5 3.1 AC-6.

\begin{displayquote}
LEAST PRIVILEGE

Control: Employ the principle of least privilege, allowing only authorized accesses for users (or
processes acting on behalf of users) that are necessary to accomplish assigned organizational
tasks.

Discussion: Organizations employ least privilege for specific duties and systems. The principle of
least privilege is also applied to system processes, ensuring that the processes have access to
systems and operate at privilege levels no higher than necessary to accomplish organizational
missions or business functions. Organizations consider the creation of additional processes, roles,
and accounts as necessary to achieve least privilege. Organizations apply least privilege to the
development, implementation, and operation of organizational systems.
\end{displayquote}

\url{https://nvlpubs.nist.gov/nistpubs/SpecialPublications/NIST.SP.800-53r5.pdf#%5B%7B%22num%22%3A181%2C%22gen%22%3A0%7D%2C%7B%22name%22%3A%22XYZ%22%7D%2C88%2C336%2C0%5D}

Resouce Owners

All resources should have one or more "owners" for the proposes of access control. In order for a user to be granted access to a resource an owner will need to sign-off on that request.  resource owners are also responsible for auditing the list of authorized users for resource and filing requests to remove access when a user no longer requires use of the resource

General workflow

- open an IHS ticket and request access to X resource
- A resource owner must sign off on the ticket before access may be granted

\appendix
% Include all the relevant bib files.
% https://lsst-texmf.lsst.io/lsstdoc.html#bibliographies
\section{References} \label{sec:bib}
\renewcommand{\refname}{} % Suppress default Bibliography section
\bibliography{local,lsst,lsst-dm,refs_ads,refs,books}

% Make sure lsst-texmf/bin/generateAcronyms.py is in your path
\section{Acronyms} \label{sec:acronyms}
\input{acronyms.tex}
% If you want glossary uncomment below -- comment out the two lines above
%\printglossaries





\end{document}
